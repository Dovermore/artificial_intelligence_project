% BEGIN AUTOMATICALLY GENERATED TIKZ DIAGRAM
\documentclass{article}

\usepackage{tikz}

\begin{document}

\begin{center} % change X in `scale=X` to resize picture (spec uses 0.35-0.8)
\begin{tikzpicture}[scale=0.5]%<---'
    % draw blank board
    \foreach \q/\r in {0/-3,1/-3,2/-3,3/-3,
                     -1/-2,0/-2,1/-2,2/-2,3/-2,
                  -2/-1,-1/-1,0/-1,1/-1,2/-1,3/-1,
                -3/0, -2/0, -1/0, 0/0, 1/0, 2/0, 3/0,
                   -3/1, -2/1, -1/1, 0/1, 1/1, 2/1,
                     -3/2, -2/2, -1/2, 0/2, 1/2,
                       -3/3, -2/3, -1/3, 0/3} {
        % draw a single blank hex,
        \draw[fill=black!3!white]
            (2*\q*sin{60}+\r*sin{60},-\r*1.5) % starting coordinate
            +(-30:1)                          % use relative angular coords (θ,r)
            \foreach \x in {30,90,...,330}    % loop through vertex angles
                { -- +(\x:1)};                % connect an edge to vertex
    }

    % draw blocks
    \foreach \q/\r in {SUBSTITUTE_BLOCKS}
        % draw a single block in this hex (like above, but smaller and darker)
        \draw[fill=black!30!white]
            (2*\q*sin{60}+\r*sin{60},-\r*1.5)
            +(-30:0.8)
            \foreach \x in {30,90,...,330}
                { -- +(\x:0.8)};
    
    % Draw the pieces too:   
    \foreach \q/\r/\c in {SUBSTITUTE_PIECES}
        % draw a single piece in this hex (a while circle ad the calculated pos)
        \node[draw,shape=circle,fill=white,inner sep=3pt]
            at (2*\q*sin{60}+\r*sin{60},-\r*1.5) %   ^     
            {$\c$};                              %   |    <-- labelled R, G or B
                                                 %   `change X in `inner sep=Xpt`
                                                 %    to get circle size right
                                                 %    (depends on scale; 0.8-5)
\end{tikzpicture}
\end{center}

\end{document}
% END AUTOMATICALLY CREATED TIKZ DIAGRAM